\part{Bevezetés}

Az Archytex egy 3D-s építészeti tervező és látványterv készítő alakmazás.

Fő célja, hogy egy olyan felületet biztosítson mind professzionális és hobbi szintű felhasználóknak, amely egy építési projekt tervezési folyamatának kezdetétől fogva egészen a látványterv bemutatásáig használható.

Jelentős hangsúly került az optimalizációra, mivel elsődleges szempont volt, hogy gyengébb rendszereken is zökkenőmentesen használható legyen az alkalmazás. Ennek eredményeként született meg a böngészőben futó 3D szerkesztő és a szerveroldali elosztott rendszerű sugárkövetéses renderelőmotor. Így a felhasználó számítógépének erőforrásainak teljes felemésztése nélkül is lehetővé válik az élethű látványtervek készítése.

A letisztult felhasználói kezelőfelület is a könnyű használhatóságot segíti. Regisztráció után bárki hozhat létre projektet, amelyben a beépíett eszközök segítségével tervezhet, majd a textúrákat és bútorokat felhasználva életre kelheti építészeti ötleteit. Az alkalmazás könnyedén testreszabható: egy gombnyomással váltható a sötét és világos mód, és a felület számos nyelven elérhető.

Számos olyan alkalmazás létezik, amely építészeti tervezésre vagy látványtervek készítésére szakosodik, de az Archytex-el szemben egyik sem biztosítja az összes funkcionalitást a tervezéstől egészen a látványterv elkészítéséig egy csomagban. Az Archytex használatához nem szülséges semmit letölteni, az összes mondern böngészőben futtatható, mivel az egész alkalmalmazás webes technológiákra épül.

A feljlesztési folyamat során motivációt vettünk több, már létező programból, mint például a Valve Hammer Editor\footnote{\url{https://developer.valvesoftware.com/}} (objektumok és textúrák megjelenítése), a Blender\footnote{\url{https://www.blender.org/}} (objektum transzformációk billentyűkombinációi) vagy az Unreal Engine\footnote{\url{https://www.unrealengine.com/}} (navigáció a szerkesztő 3D terében). Ezen programok funkciója és célközönsége nagyban különbözik az Archytex-étől, viszont egyes elemeket és ötleteket átemelve kiválóan kiegészítik alkalmazásunk funkcionalitását.