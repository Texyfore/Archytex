\section{Szerkesztő}
\rhead{Szerkesztő}

\subsection{A szerkesztő célja}

Az Archytex szerkesztő célja egy eszköz biztosítása építészeti látványtervek elkészítésére. A kész
jelenetek a beépített \emph{renderelés} funckió segítségével élethű képekké, professzionális
építészeti vizualizációkká alakíthatók, amelyek használhatók hobbi szinten, portfólió építésére vagy
akár professzionális munkához.

\subsection{Térbeli navigáció}

A szerkesztő legalapvetőbb eszköze a mozgatható kamera, hisz nélküle egy másik eszközt sem lehet
hasznosan alkalmazni. Szerencsére működése nagyon egyszerűen elsajátítható. Ahhoz, hogy a kamera
mozogjon, nyomva kell tartani a \emph{jobb egérgombot} egészen addig, amíg mozogni szeretnénk.
A \emph{jobb egérgomb} elengedésekor a kamera kilép a mozgó üzemmódból. Miközben a kamera mozgó
üzemmódban van, előre, hátra, balra és jobbra a \emph{W, S, A és D}, míg le és
fel az \emph{E és Q} billentyűkkel mozog. A kamera célirányának változtatása az egér mozgatásával
érhető el.

\subsection{Általános műveletek}

Létezik több olyan művelet, ami annyira általános, hogy bármilyen kontextusból elérhető.

\subsubsection{Visszavonás}

Bármilyen műveletet vissza lehet vonni a \emph{Ctrl+Z} billentyűkombinációval. Ilyenkor a
szerkesztő eltárolja a visszavont műveletet, hogy azt később meg lehessen ismételni. Ha az eltárolt
műveletek mennyisége eléri a hatvannégyet, a legrégebbi törlésre kerül.

\subsubsection{Ismétlés}

Ha a visszavont műveletek listája nem üres, akkor a \emph{Ctrl+Y} billentyűkombinációval kiadott
\emph{ismétlés} művelet megismétli a legújabban visszavont műveletet.

\subsubsection{A rács nagyítása és kicsinyítése}

A szerkesztőben minden elem pozíciója egy négyzetes rácshoz igazodik. Az alapértelmezett rács
oldalhossza \emph{1 méter}, viszont gyakran előfordul, hogy ez túl nagy, vagy éppen túl kicsi.
Ilyenkor az \emph{O} billentyű lenyomásával a rács oldalhossza felére csökken, míg a \emph{P}
billentyű lenyomásával kétszeresére nő.

\subsubsection{Kijelölés}

A jelenet bármely eleme kijelölhető. Egy elem kijelöléséhez két feltételnek kell teljesülnie: az
\emph{egérmutatónak} az elem felett kell lennie, és meg kell nyomni a \emph{bal egérgombot}. Ha ez
úgy történik meg, hogy már léteznek kijelölt elemek, akkor azok a kijelölések megszűnnek. Ha az
előző kijelölés \emph{bővítése} a cél, akkor a \emph{Shift} billentyűt is nyomva kell tartani.
Érdemes megjegyezni, hogy ha ilyenkor rákattintunk egy kijelölt elemre, az a kijelölés megszűnik,
viszont a többi megmarad.

\subsubsection{Mozgatás}

A jelenet bármely kijelölt eleme mozgatható. Olyan esetben, amikor a mozgatás engedélyezett,
látható a képernyőn a \emph{speciális mozgatóeszköz}. Az eszköz három színezett nyílból és három
színezett négyzetből áll. A nyilak a tér három tengelye, míg a négyzetek a tér három síkja mentén
mozgatják a kijelölt elemeket. Használathoz egyszerűen \emph{rá kell kattintani} a nyílra vagy
négyzetre. Az elemek abba az irányba fognak elmozdulni, amerre az \emph{egérmutató} mozog. A
\emph{bal egérgombot} addig kell lenyomva tartani, amíg az elemek nem kerültek a megfelelő helyükre.

\subsubsection{Gyors mozgatás}

A kijelölt elemeket nem csak a \emph{speciális mozgatóeszköz} segítségével lehet mozgatni. Létezik
egy ún. \emph{gyors mozgatás} eszköz is, amit teljes mértékben billentyűkombinációk és egérmozgatás
vezérel. A \emph{gyors mozgatás} aktiválásához meg kell nyomni a \emph{G} billentyűt. Miután ez
megtörtént, ki kell választani, hogy melyik tengely vagy sík mentén mozogjanak az elemek. Az
\emph{X, Y vagy Z} billentyű lenyomása kiválasztja a hozzátartozó tengelyt. Sík kiválasztása is
hasonló módon történik: a \emph{Shift} billentyű lenyomása közben az \emph{X, Y, vagy Z} billentyű
a \emph{hozzátartozó tengelyre merőleges} síkot választja ki. Kiválasztás után az elemek az
\emph{egérmutatóval} mozognak. Ha az elemek elérték a megfelelő pozíciót, a \emph{bal egérgomb}
lenyomásával befejezbető a művelet. Abban az esetben, ha kiderül, hogy a mozgatás mégsem célszerű,
a \emph{G, Escape} billentyűk, vagy \emph{jobb egérgomb} lenyomásával visszamondható a mozgatás.

\subsection{Test szerkesztési mód}

Az általános műveletek mellett a szerkesztő támogat ún. \emph{speciális műveleteket} is, amik csak
bizonyos \emph{módokban} használhatók. A szerkesztőnek négy különböző \emph{módja} van, ezek közül
az első -- és egyben alapértelmezett -- a \emph{test szerkesztési mód}.

\subsubsection{Létrehozás}

\subsubsection{Törlés}

\subsubsection{Forgatás}

\subsubsection{Másolás}

\subsubsection{Üregesítés}

\subsection{Lap szerkeszési mód}

\subsubsection{Textúrázás}

\subsection{Csúcs szerkesztési mód}

\subsection{Berendező mód}

\subsubsection{Forgatás}

\subsubsection{Gyors forgatás}

\subsubsection{Másolás}

\subsection{Mentés és renderelés}