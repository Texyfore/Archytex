\section{Vezérlőpult}
\rhead{Vezérlőpult}


\subsection{Projektek}
A vezérlőpulton találhatók a bejelentkezett felhasználó projektjei. Lehetőség van új projekt létrehozására vagy már létező projekt átnevezésére és törlésére a projekt listaelem jobb oldalán található gombbal megnyitható menüben. Emellett megtekinthetők a projektek alábbi tulajdonságai is:
\begin{samepage}
  \begin{itemize}
    \item Projekt neve
    \item Renderek száma
    \item Létrehozás dátuma
  \end{itemize}
\end{samepage}

\subsection{Renderek}
A szerkesztőben elindított renderek láthatók a projekre kattintás utáni lenyíló kártya listában. Az újonnan elindított renderek egyből megjelennek a vezérlőpultban.

A render kártyán látható a jelenlegi készenléti százalék, amely visszajelzést ad, hogy nagyjából meddig fog tartani a renderelés. Ha a render már készen van, akkor kattinthatóvá válik a képe. A képre kattintással megtekinthető a teljes felbontású render. A kártyán található még a letöltés, a részletek és a törlés gomb. A render megjeleníthető részletei:
\begin{samepage}
  \begin{itemize}
    \item Render neve
    \item Státusz (\%)
    \item Indítás dátuma
    \item Befejezés dátuma
  \end{itemize}
\end{samepage}

\subsection{Beállítások}
A beállítások lap a navigációs csík jobb oldalán lévő profil ikonra kattintva megjelenő menüben található \emph{Beállítások} gombbal érhető el. Ezen az oldalon megváltoztathatók a felhasználó adatai, valamint lehet váltani sötét és világos mód között.