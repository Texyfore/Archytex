\section{Backend}
\rhead{Backend}
\subsection{Használt technológiák}
A backend megírásánál az egyszerűségre és a későbbi könnyebb felhasználhatóságra törekedtünk, így esett választásunk a Go-MongoDB stack-re.
\subsubsection{MongoDB}
A \emph{MongoDB}\footnote{\url{https://www.mongodb.com/}} egy NoSQL adatbázis, amely annyit tesz, hogy táblák és sorok helyett egy JSON szerű struktúrában tárolja el az adatokat.
Ennek több előnye és hátránya is van.

\begin{samepage}
\noindent Előnyei:
\begin{itemize}
    \item Egyszerű feltöltés/letöltés a JSON szerű formátum miatt
    \item Nem szükséges előre felépíteni az adatbázis struktúráját
    \item Frissítések valós-idejű figyelése
\end{itemize}

\noindent Hátrányai:
\begin{itemize}
    \item Ritkábban használt
    \item Átlagoshoz képest komplexebb lekérdezések
    \item Sémának való megfelelés nem ellenőrzött
    \item Kevésbé rendezett, mint a relációs adatbázisok
\end{itemize}
\end{samepage}

Az említett valós-idejű figyelés, amely a projekt frissítéséhez szükséges, csak replica set esetén működik, tehát akkor ha a szerver be van állítva a megosztott munkára.
\subsubsection{Go}
A stack másik tagja, a \emph{Go}\footnote{\url{https://golang.org/}} egy programozás nyelv, amelyet a Google fejlesztett ki. A cél egy átlátható, egyszerű nyelv készítése volt, amelyet a \emph{Python}-t használó fejlesztők is könnyen megtanulhatnak. \todo[inline]{Citation: Tényleg a python developerek miatt készült?}

\begin{samepage}
\noindent Előnyei:
\begin{itemize}
    \item Típusos, így kisebb az esélye a hibáknak
    \item Erős beépített web fejlesztési (HTTP, sablonok stb.) könyvtárak
    \item Egyszerű párhuzamos futtatás \emph{Goroutine}-okkal és csatornákkal
    \item Fordított, de szemétgyűjtős
    
\end{itemize}
\end{samepage}

\begin{samepage}
\noindent Hátrányai:
\begin{itemize}
    \item Adatfeldolgozási függvényekben hiányos
    \item Generikusok hiánya
    \item Frusztrálóan bőszavú hibakezelés
\end{itemize}
\end{samepage}

Az e-mail rendszerhez a beépített net/smtp \todo{Acronym} könyvtárat használtuk az üzenetek elküldéséhez, illetve a html/template könyvtárat az üzenetek tartalmának kitöltéséhez.

A fejlesztésben sokat segített a \emph{Gorilla} könyvtár, amely az alapértelmezett HTTP könyvtárat egészíti ki további hasznos funkcióval.
Például middleware támogatással, metódus korlátozással, URL-be helyezett paraméterekkel stb.

\subsection{Szolgáltatások méretezése}
A \emph{docker-compose} fájlban található \emph{Traefik}\footnote{\url{https://traefik.io/traefik/}} két fő feladatot lát el:

\begin{samepage}
\begin{itemize}
    \item Statikus frontend fájlok és a Backend api egy végpontra helyezése
    \item Automatikus terheléselosztás
\end{itemize}
\end{samepage}

A \emph{Traefik} beintegrálja magát a \emph{Docker}-be és így látja, ha egy szolgáltatásból több fut és el tudja közöttük osztani a terhet.

Ha szeretnénk egy szolgáltatást méretezni, akkor a \emph{docker-compose} parancsot kell lefuttatnunk, a következő parancsok egyikével:

\begin{itemize}
    \item Statikus fájl szerver méretezése
    \begin{lstlisting}[language=bash]
        $ docker-compose up --scale frontend=<DARAB>
    \end{lstlisting}
    \item Backend szerver méretezése
    \begin{lstlisting}[language=bash]
        $ docker-compose up --scale backend=<DARAB>
    \end{lstlisting}
\end{itemize}
    

\todo[inline]{Adatbázis model}
\todo[inline]{HTTP dokumentáció}
\todo[inline]{WS dokumentáció}