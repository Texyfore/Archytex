\section{Backend}
\rhead{Backend}
\subsection{Használt technológiák}
A backend megírásánál az egyszerűségre és a későbbi könnyebb felhasználhatóságra törekedtünk, így esett választásunk a Go-MongoDB stack-re.
\subsubsection{MongoDB}
A \emph{MongoDB}\footnote{\url{https://www.mongodb.com/}} egy NoSQL adatbázis, amely azt jelenti, hogy táblák és sorok helyett egy JSON szerű struktúrában tárolja el az adatokat.
Ennek számos előnye és hátránya is van.

\begin{samepage}
    \noindent Előnyei:
    \begin{itemize}
        \item Egyszerű feltöltés/letöltés a JSON szerű formátum miatt
        \item Nem szükséges előre felépíteni az adatbázis struktúráját
        \item Frissítések valós-idejű figyelése
    \end{itemize}

    \noindent Hátrányai:
    \begin{itemize}
        \item Ritkábban használt
        \item Átlagoshoz képest komplexebb lekérdezések
        \item Sémának való megfelelés nem ellenőrzött
        \item Kevésbé rendezett, mint a relációs adatbázisok
    \end{itemize}
\end{samepage}

Az említett valós-idejű figyelés, amely egy felhasználó projektjeinek folyamatos frissítéséhez szükséges, csak replica set esetén működik, tehát akkor ha a szerver be van állítva a megosztott munkára.
\subsubsection{Go}
A stack másik tagja, a \emph{Go}\footnote{\url{https://golang.org/}} egy programozás nyelv, amelyet a Google fejlesztett ki. A cél egy átlátható, egyszerű nyelv készítése volt, amelyet a \emph{Python}-t használó fejlesztők is könnyen megtanulhatnak. \todo[inline]{Citation: Tényleg a python developerek miatt készült?}

\begin{samepage}
    \noindent Előnyei:
    \begin{itemize}
        \item Típusos, így kisebb az esélye a hibáknak
        \item Erős beépített webfejlesztési (HTTP, sablonok stb.) könyvtárak
        \item Egyszerű párhuzamos futtatás \emph{Goroutine}-okkal és csatornákkal
        \item Fordított, de szemétgyűjtős

    \end{itemize}
\end{samepage}

\begin{samepage}
    \noindent Hátrányai:
    \begin{itemize}
        \item Adatfeldolgozási függvényekben hiányos
        \item Generikusok hiánya
        \item Frusztrálóan bőszavú hibakezelés
    \end{itemize}
\end{samepage}

Az e-mail rendszerhez a beépített NET/SMTP \todo{Acronym} könyvtárat használtuk az üzenetek elküldéséhez, illetve a HTML/template könyvtárat az üzenetek tartalmának kitöltéséhez.

A fejlesztésben sokat segített a \emph{Gorilla} könyvtár, amely az alapértelmezett HTTP könyvtárat egészíti ki további hasznos funkciókkal.
Például middleware támogatással, metódus korlátozással, URL-be helyezett paraméterekkel stb.

\subsection{Szolgáltatások méretezése}
A \emph{docker-compose} fájlban található \emph{Traefik}\footnote{\url{https://traefik.io/traefik/}} két fő feladatot lát el:

\begin{samepage}
    \begin{itemize}
        \item Statikus frontend fájlok és a Backend api egy végpontra helyezése
        \item Automatikus terheléselosztás
    \end{itemize}
\end{samepage}

A \emph{Traefik} integrálja magát a \emph{Docker}-be, és így látja, ha egy szolgáltatásból több fut, és el tudja közöttük osztani a terhet.

Ha szeretnénk egy szolgáltatást méretezni, akkor a \emph{docker-compose} parancsot kell lefuttatnunk, a következő parancsok egyikével:

\begin{itemize}
    \item Statikus fájl szerver méretezése
          \begin{lstlisting}[language=bash]
        $ docker-compose up --scale frontend=<DARAB>\end{lstlisting}
    \item Backend szerver méretezése
          \begin{lstlisting}[language=bash]
        $ docker-compose up --scale backend=<DARAB>\end{lstlisting}
\end{itemize}

\subsection{Adatbázis modell}

Az adatokat a \emph{MongoDB} nevű adatbáziskezelőben tároljuk el, amely NoSQL technológiára épül, így teljesen más struktúrát követel, mint egy relációs adatbázis. Míg a relációs adatbázisok egy gráf szerkezetet használnak, addig a \emph{MongoDB} egy fastruktúrát.

Három különböző dokumentumtípust különböztetünk meg:
\begin{itemize}
    \item users

          A regisztrált és regisztrációjukat megerősített felhasználók listája
    \item registers

          A regisztrált, de még nem megerősített felhasználók listája
    \item sessions

          A jelenleg belépett felhasználók listája
\end{itemize}

\subsubsection{Users dokumentum}
\begin{itemize}
    \item \_id

          Típus: ObjectID

          Leírás: A \emph{MongoDB} által generált egyedi azonosító
    \item username

          Típus: string

          Leírás: A felhasználó választott neve
    \item email

          Típus: string

          Leírás: A felhasználó e-mail címe
    \item password

          Típus: string

          Leírás: A felhasználó jelszavának ellenőrzésére szolgáló kriptográfiai karakterlánc
    \item projects

          Típus: Tömb, Lásd lent

          Leírás: A felhasználó projektjeinek listája
\end{itemize}

\subsubsection{Projects al-dokumentum}

\begin{itemize}
    \item \_id

          Típus: ObjectID

          Leírás: A \emph{MongoDB} által generált egyedi azonosító
    \item title

          Típus: string

          Leírás: A projekt címe
    \item created

          Típus: date

          Leírás: A projekt létrehozásának időpontja
    \item path

          Típus: string

          Leírás: A szerver fájlrendszerében lévő ASCN fájl neve
    \item renders

          Típus: Tömb, Lásd lent

          Leírás: A projekthez tartozó renderek listája
\end{itemize}
\subsubsection{Renders al-al-dokumentum}

\begin{itemize}
    \item \_id

          Típus: ObjectID

          Leírás: A \emph{MongoDB} által generált egyedi azonosító
    \item name

          Típus: string

          Leírás: A render neve
    \item status

          Típus: double

          Leírás: A render elkészültségének jelenlegi állapota, 0-tól 1-ig tartó valós szám
    \item started

          Típus: date

          Leírás: A render munka elküldésének időpontja
    \item finished

          Típus: date

          Leírás: A render befejezésének időpontja
    \item icon

          Típus: string

          Leírás: A szerver fájlrendszerében lévő képfájl neve
\end{itemize}

\subsubsection{Registers dokumentum}

\begin{itemize}
    \item \_id

          Típus: ObjectID

          Leírás: A \emph{MongoDB} által generált egyedi azonosító
    \item username

          Típus: string

          Leírás: A felhasználó választott neve
    \item email

          Típus: string

          Leírás: A felhasználó ímélcíme
    \item password

          Típus: string

          Leírás: A felhasználó jelszavának ellenőrzésére szolgáló kriptográfiai karakterlánc
    \item token

          Típus: string

          Leírás: A regisztáció megerősítéséhez szükséges egyszer használatos azonosító
    \item created\_at

          Típus: date

          Leírás: A regisztrációs szándék beadásának időpontja. Segítségével létre lehez hozni olyan indexet, ami a kérelmet törli bizonyos idő lejárta után.
\end{itemize}
\subsubsection{Sessions dokumentum}

\begin{itemize}
    \item \_id

          Típus: STRING!!!

          Leírás: Kriptográfiailag generált karakterlánc, amely a felhasználót azonosítja bejelentkezés után
    \item user\_id

          Típus: ObjectID

          Leírás: A felhasználó egyedi azonosítója
\end{itemize}

\todo[inline]{Adatbázis struktúra berakása}

\subsection{Backend .env fájl}

A backend konfigurálását egy .env fájl segítségével lehet megtenni. Ha a Docker környezetet használjuk, ezt a fájlt nem szükséges kitölteni, ahhoz külön .env fájl tartozik.

Szükséges konfigurációs paraméterek:
\begin{itemize}
    \item SMTP beállítások
          \subitem SMTP\_SERVER

          SMTP szerver címe
          \subitem SMTP\_ADDRESS

          SMTP szerveren lévő ímélcím
          \subitem SMTP\_PASSWORD

          Az előbb használt ímélcímhez tartozó jelszó
          \subitem SMTP\_PASSWORD

          Az előbb használt ímélcímhez tartozó jelszó
    \item MongoDB beállítások
          \subitem MONGO\_URI

          MongoDB-hez való csatlakozáshoz szükséges URI
          \subitem MONGO\_DB

          MongoDB adatbázis neve
    \item RabbitMQ beállítások
          \subitem AMQP\_ADDR

          AMQP (RabbitMQ)-hez való csatlakozáshoz szükséges URI
    \item Redis beállítások
          \subitem REDIS\_ADDR

          Redis-hez való csatlakozáshoz szükséges URI
    \item Egyéb
          \subitem PROJECTS\_PATH

          Projektfájlok és renderek tárolására elkülönített mappa elérési útja
          \subitem PORT

          Szerver portja
          \subitem DOMAIN

          A szerver távoli címe, amelyet a megerősítő ímélben használunk
          \subitem CAPTCHA\_SECRET

          ReCaptcha szolgáltatásból érkező titkos kulcs
\end{itemize}

Példa:

\begin{lstlisting}
PORT=8080
MONGO_URI=mongodb://localhost:27017
MONGO_DB=archytex
SMTP_SERVER=smtp.gmail.com:587
SMTP_ADDRESS=archytex@gmail.com
SMTP_PASSWORD=password
AMQP_ADDR=amqp://127.0.0.1:5672
REDIS_ADDR=127.0.0.1:6379
CAPTCHA_SECRET=SECRETSECRETSECRETSECRETSECRETSECRET
PROJECTS_PATH=F:/archytex_storage
DOMAIN=https://archytex.me
\end{lstlisting}

\todo[inline]{HTTP dokumentáció}
\todo[inline]{WS dokumentáció}