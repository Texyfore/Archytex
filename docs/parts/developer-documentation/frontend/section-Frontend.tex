\section{Frontend}
\rhead{Frontend}

A frontend a szoftver azon része, amely nagyban meghatározza egy felhasználó élményét. Fontos az átláthatóság, a könnyű kezelhetőség és az esztétika. Ezen szempontokat figyelembe véve készítettük el az Archytex weboldalt.

\subsection{Design terv}
A honlap elkészítésének első lépése a tervezés volt. Ehhez a Figma\footnote{\url{https://www.figma.com/}} nevű ingyenes design szoftvert használtuk. Ezen alkalmazás segítségével könnyedén tudtunk vázlatot készíteni arról, hogy hogyan képzeljük el a honlap megjelenését, még azelőtt, hogy elkezdtük volna a programozást.

A Figma a design tervezés mellett használható vektor grafikák rajzolására is, amellyel az Archytex szerkesztőben látható ikonok készültek.

\begin{figure}[h]
  \centering
  \includegraphics[width=0.1\textwidth]{parts/developer-documentation/frontend/images/meshSelectMode.png}
  \includegraphics[width=0.1\textwidth]{parts/developer-documentation/frontend/images/faceSelectMode.png}
  \includegraphics[width=0.1\textwidth]{parts/developer-documentation/frontend/images/vertexSelectMode.png}
  \caption{Példa a Figma-ban készíthető vektor grafikákra: a kiválasztási módok ikonjai az Archytex szerkesztőből.}
\end{figure}

A honlapon megjelenő grafikák egy része az Undraw\footnote{\url{https://undraw.co/}} nevű honlapon található, ingyenesen használható illusztrációk felhasználásával készült. Ezen illusztrációk nyílt licensszel rendelkeznek, ezért szerkeszthetőek és ingyenesen felhasználhatóak. Néhány grafikát a Figma-ban szerkesztettünk, hogy több építészeti elemet tartalmazzanak, vagy hogy jobban illeszkedjenek a design környezetbe.

Az Archytex logo is a Figma beépített vektor grafikai eszközeivel készült. A design követi a letisztultság elvét, és használja a honlap elsődleges színét. A logó sötét és világos módtól függően a honlapon megváltozik a jobb láthatóság érdekében.


\subsection{Használt technológiák}
A frontend alkalmazás React-ben\footnote{\url{https://hu.reactjs.org/}} készült. Azért esett a választás erre a keretrendszerre, mert a horog alapú komponens állapot- és és életcikluskezelése miatt fejlesztői élményét tekintve kiemelkedően jobb, mint más rendszerek. Emellett az is segítette a választást, hogy jelenleg a React a legnépszerűbb a webfejlesztők köreiben\cite{most-used-web-frameworks}, ezért rengeteg forrás és oktatóvideó érhető el hozzá.

A honlap elkészítéséhez a Material UI\footnote{\url{https://mui.com/}} nevű komponens könyvtárat használtuk, amely rengeteg előre elkészített és könnyen testreszabható komponenst tartalmaz. Ennek segítségével és a Google által kifejlesztett Material Design\footnote{\url{https://material.io/}} irányelveit követve modern, letisztult és felhasználóbarát felületet tudtunk létrehozni.

\subsection{Főoldal}
A főoldal célja, hogy egy új felhasználónak "eladja" a termékünket. Ezért fontos volt, hogy látványos és emlékezetes legyen, de ezek mellett egyben informatív is. A látványossághoz nagyban hozzájárultak az Undraw.io-s illusztrációk és a tsparticles\footnote{\url{https://github.com/matteobruni/tsparticles/tree/main/components/react}} JavaScript könyvtár segítségével létrehozott, az oldal fejlécében megjelenő interaktív buborékok. Emellett az interaktívitás növelése érdekében az AOS (Animate On Scroll) könyvtárat használva megoldottuk, hogy a honlap tartalma folyamatosan jelenjen meg, ahogy a felhasználó görget lefelé a honlapon.

\begin{figure}[h]
  \centering
  \includegraphics[width=\textwidth]{parts/developer-documentation/frontend/images/header.png}
  \caption{Az Archytex főoldal fejléce}
\end{figure}

\subsection{Autentikáció}
A bejelentkezési és regisztrációs képernyő frontend funkcionalitás szempontjából egy viszonylag nagyobb kihívást nyújtott, mint más oldalak. Tudtuk, hogy gyakran lesz szükségünk olyan űrlapokra a projekt folyamán, amiben a felhasználó felé visszajelzést kell küldeni, ezért készítettünk egy generalizált React komponenst ennek a feladatnak az ellátására. Ezzel beviteli mező komopnenssel már könnyedén fel tudtunk építeni mind a bejelentkezési és a regisztrációs űrlapot, és a hibakezelés sem okozott problémát.

\begin{figure}[h]
  \centering
  \begin{minipage}{.5\textwidth}
    \centering
    \includegraphics[width=.6\linewidth]{parts/developer-documentation/frontend/images/login.png}
    \label{fig:test1}
  \end{minipage}%
  \begin{minipage}{.5\textwidth}
    \centering
    \begin{lstlisting}
      <FormInput
        variant='username'
        label={t("username")}
        input={username}
        inputChange={handleUsernameChange}
        error={usernameError}
      />
    \end{lstlisting}
  \end{minipage}
  \caption{Példa az egyedi beviteli komponens használatára a bejelentkezési képernyőn}
\end{figure}

\subsection{Irányítópult}
Az Archytex irányítópult feladata a projektek listázása és egy felhasználói interfész biztosítása ezen projektek beállításainak módosítására, tulajdonságaik megtekintésére.

\subsection{Szekesztő UI}

\subsection{Tesztek}