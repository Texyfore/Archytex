\section{Frontend}
\rhead{Frontend}

A frontend a szoftver azon része, amely nagyban meghatározza egy felhasználó élményét. Fontos az átláthatóság, a könnyű kezelhetőség és az esztétika. Ezen szempontokat figyelembe véve készítettük el az Archytex weboldalt.

\subsection{Design terv}
A honlap elkészítésének első lépése a tervezés volt. Ehhez a Figma\footnote{\url{https://www.figma.com/}} nevű ingyenes design szoftvert használtuk. Ezen alkalmazás segítségével könnyedén tudtunk vázlatot készíteni arról, hogy hogyan képzeljük el a honlap megjelenését, még azelőtt, hogy elkezdtük volna a programozást.

A Figma a design tervezés mellett használható vektor grafikák rajzolására is, amellyel az Archytex szerkesztőben látható ikonok készültek.

% TODO: Undraw.io
Undraw.io

Logo design

\begin{figure}
  \centering
  \includegraphics[width=0.15\textwidth]{parts/developer-documentation/frontend/images/meshSelectMode.png}
  \includegraphics[width=0.15\textwidth]{parts/developer-documentation/frontend/images/faceSelectMode.png}
  \includegraphics[width=0.15\textwidth]{parts/developer-documentation/frontend/images/vertexSelectMode.png}
\end{figure}

\subsection{Használt technológiák}
A frontend alkalmazás React-ben\footnote{\url{https://hu.reactjs.org/}} készült. Azért esett a választás erre a keretrendszerre, mert a horog alapú komponens állapot- és és életcikluskezelése miatt fejlesztői élményét tekintve kiemelkedően jobb, mint más rendszerek. Emellett az is segítette a választást, hogy jelenleg a React a legnépszerűbb a webfejlesztők köreiben\cite{most-used-web-frameworks}, ezért rengeteg forrás és oktatóvideó érhető el hozzá.

A honlap elkészítéséhez a Material UI\footnote{\url{https://mui.com/}} nevű komponens könyvtárat használtuk, amely rengeteg előre elkészített és könnyen testreszabható komponenst tartalmaz. Ennek segítségével és a Google által kifejlesztett Material Design\footnote{\url{https://material.io/}} irányelveit követve modern, letisztult és felhasználóbarát felületet tudtunk létrehozni.

\subsection{Főoldal}
A főoldal célja, hogy egy új felhasználónak "eladja" a termékünket. Ezért fontos volt, hogy látványos és emlékezetes legyen, de ezek mellett egyben informatív is. A látványossághoz nagyban hozzájárultak az Undraw.io-s illusztrációk és a tsparticles\footnote{\url{https://github.com/matteobruni/tsparticles/tree/main/components/react}} JavaScript könyvtár segítségével létrehozott, az oldal fejlécében megjelenő interaktív buborékok. Emellett

\subsection{Autentikáció}
\subsection{Irányítópult}
\subsection{Editor}
\subsection{Tesztek}