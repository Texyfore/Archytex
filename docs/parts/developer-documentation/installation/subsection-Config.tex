\subsection{.env fájl létrehozása}
\label{config}

A szerverek konfigurálásához egy .env fájlt létrehozására van szükség. A repository törzsében található egy .env.sample fájl, amely egy példa konfiguráció.

Másolja ezt a .env.sample fájlt és nyissa meg egy szövegszerkesztő programmal, például Jegyzettömbbel.

A fájl minden sorában egy-egy konfigurációs elem található. Az elemek egy névből és egy értékből állnak.

A DOMAIN konfigurációhoz a szolgáltatás címét kell megadni, például: archytex.tech

Az SMTP kezdetű sorok szükségesek a regisztrációt megerősítő ímélek elküldéséhez. A pontos értékekhez keresse meg e-mail szolgáltatás közvetítőjének dokumentációját.

A CAPTCHA kezdetű sorok a bejelentkezéshez szükséges CAPTCHA-t konfigurálják. Ehhez szükség van egy ReCaptcha V2 kulcspárra, amit a ReCaptcha oldalon\footnote{\url{https://www.google.com/recaptcha/admin/}} hozhat létre. A CAPTCHA\_SECRET után a \emph{secret key}-t kell rakni, míg a CAPTCHA\_PUBLIC után a \emph{site key}-t.

Végül szükséges egy jelszó generálása, amelyet a REDIS\_PASSWORD részbe kell elhelyezni. Ez a jelszó használható a későbbi csatlakozáshoz.