\subsection{.env fájl létrehozása}
\label{config}

A szerverek konfigurálásához létre kell hozni egy .env fájlt. A repository törzsében található egy .env.sample fájl. Ez egy példa konfiguráció.

Másolja ezt a .env.sample fájlt és nyissa meg egy szövegszerkesztő programmal, például Jegyzettömbbel.

A fájlban minden sorba egy-egy konfigurációs elem található. Az elemek egy névből és egy értékből állnak.

A DOMAIN konfigurációhoz a szolgáltatás címét kell megadni, például: archytex.tech

Az SMTP kezdetű sorok szükségesek a regisztrációt megerősítő ímélek elküldéséhez. A pontos értékekhez keresse meg ímélszolgáltatás követítőjének dokumentációját.

A CAPTCHA kezdetű sorok a bejelentkezéshez szükséges Captchát konfigurálják. Ehhez szükség van egy ReCaptcha V2 kulcspárra, amit a ReCaptcha oldalon \footnote{\url{https://www.google.com/recaptcha/admin/}} hozhat létre. A CAPTCHA\_SECRET után a \emph{secret key}-t kell rakni, míg a CAPTCHA\_PUBLIC után a \emph{site key}-t.

Végül, a REDIS\_PASSWORD részbe generálnia kell egy jelszót, amely majd később a csatlakozáshoz lesz használva.