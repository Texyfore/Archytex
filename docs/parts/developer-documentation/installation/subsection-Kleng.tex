\subsection{Erőforrások létrehozása}

A program lehetőséget nyújt arra, hogy az üzemeltető a beépített erőforrásokat
(textúrák, díszítőelemek) kicserélje sajátokra. Ha erre nincs igény, akkor ezt a lépést ki lehet
hagyni.

\subsubsection{Előkészületek}

\begin{enumerate}
    \item Hozzon létre egy tetszőleges nevű könyvtárat. A továbbiakban a dokumentum erre a
    könyvtárra \emph{assets} néven hivatkozik.

    \item Hozzon létre két új könyvtárat az \emph{assets} könvytárban. Az egyik neve legyen
    \emph{textures}, a másiké legyen \emph{props}.

    \item Hozzon létre egy új üres fájlt az \emph{assets} könyvtárban. A neve legyen \emph{assets},
    a kiterjesztése legyen \emph{json}.
\end{enumerate}

\subsubsection{Textúrák elhelyezése}

Gyűjtse össze az összes használni kívánt textúrát, és helyezze őket az \emph{assets/textures}
könyvtárba. A textúrák kizárólag PNG képek lehetnek. Nem követelmény, de a helyes működés érdekében
ajánlott, hogy minden textúra legyen 1024 pixel széles és 1024 pixel magas. Megjegyzés: Azok a
textúrák is idetartoznak, amelyeket csak díszítőelemek használnak, és felhasználó számára nem
publikusak.

\subsubsection{Díszítőelemek elhelyezése}

Gyűjtse össze az összes használni kívánt díszítőelemet, és helyezze őket az \emph{assets/props}
könyvtárba. A díszítőelemek kizárólag beágyazott GLTF modellek lehetnek. A modellnek egyedi
névvel rendelkező almodellekből kell állnia. Minden almodellhez egyetlen textúra tartozhat.

\pagebreak

\subsubsection{Az \emph{assets.json} fájl kitöltése}

Az \emph{assets.json} fájl tartalmazza az összes textúra és díszítőelem tulajdonságait. Töltse ki
a fájlt a következő séma szerint:

\begin{figure}[h]
\centering
\begin{minipage}{.7\textwidth}
\centering
\begin{lstlisting}
{
    "textures": {
        "textura_neve": {
            "diffuse": "kep_neve.png",
            "emissive": "emissziv_kep_neve.png" | null,
            "categories": ["kategoria1", "kategoria2", ...]
        },
        ...
    },
    "props": {
        "diszitoelem_neve": {
            "source": "modell_neve.gltf",
            "categories": ["kategoria1", "kategoria2", ...],
            "textures": {
                "AlmodellNeve": "textura_neve"
            }
        },
        ...
    }
}
\end{lstlisting}
\end{minipage}
\end{figure}

Megjegyzés: Textúrának csak akkor kell kategóriákat adni, ha azt szeretné, hogy megjelenjen a
textúrakönyvtárban. Ellenkező esetben a \emph{categories} tömb maradjon üresen.

\pagebreak

\subsubsection{A kleng telepítése}

Ahhoz, hogy az eddig összegyűjtött erőforrásokat a szerkesztő használni tudja, először át kell őket
konvertálni.

Az Archytex GitHub repository-ja tartalmazza a \emph{kleng} nevű program forrását. Ez a program
felelős az erőforrások használható formátummá konvertálásáért. Ahhoz, hogy továbbléphessen, ezt a
programot fel kell telepítenie. A telepítéshez szüksége lesz a
\emph{cargo}\footnote{https://doc.rust-lang.org/cargo/getting-started/installation.html}
programra. Miután feltelepítette a \emph{cargo}-t a számítógépére, nyisson meg egy parancssort
(Windows) vagy terminált (Linux, MacOS), és adja ki a következő parancsot:

\begin{figure}[h]
\centering
\begin{minipage}{.7\textwidth}
\begin{lstlisting}
cargo install --path (ARCHYTEX_HELYE)/kleng
\end{lstlisting}
\end{minipage}
\end{figure}

A folyamat eltarthat néhány percig. Miután befejeződött, navigáljon el az \emph{assets} könyvtárba,
és adja ki a következő parancsot:

\begin{figure}[h]
\centering
\begin{minipage}{.7\textwidth}
\begin{lstlisting}
kleng .
\end{lstlisting}
\end{minipage}
\end{figure}

Ha nincs hiba az összegyűjtött erőforrásokban vagy az \emph{assets.json} fájlban, a program
csendben kilép. Ha hibaüzenet jelenik meg, ellenőrizze az erőforrások helyességét és próbálja
újra. Sikeres konvertálás esetén az \emph{assets} könyvtárban megjelenik egy \emph{out}
nevű könyvtár.

\subsubsection{Erőforrások elhelyezése}

Végül a \emph{kleng} által generált állományokat át kell másolni a megfelelő helyekre.

\begin{enumerate}
    \item Az \emph{out/public} könyvtár tartalmát másolja át az \\
    \emph{(ARCHYTEX\_HELYE)/frontend/public/assets} könyvtárba.

    \item Az \emph{out/repo.json} fájlt másolja át az \\
    \emph{(ARCHYTEX\_HELYE)/frontend/public/assets} könyvtárba.

    

\end{enumerate}