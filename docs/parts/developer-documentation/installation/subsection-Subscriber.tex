\subsection{Raytracer feliratkozó beállítása}\label{subscriber}


A Raytracer funkció működése érdekében szükséges létrehozni egy Feliratkozó szervert.
Ehhez a GitHub oldalon a \emph{Releases} menüpont alatt találja meg a futtatható bináris fájlt.
Ez a bináris fájl a \emph{raytracer/archyrt-sub} útvonalon található Cargo crate-ből jött létre.


\begin{enumerate}
    \item Hozzon létre egy mappát a feliratkozónak (pl. \emph{subscriber} néven), amelyben hozzon létre még egy \emph{assets} és egy \emph{renderer} nevű mappát.

    \item A \emph{renderer} mappába másolja a bináris fájlt. Helyezze az \emph{assets} mappába a \emph{kleng} által generált, vagy a már kész erőforráscsomagban található raytracer minőségű erőforrásokat, illetve a repo.json mappát.

    \item A \emph{renderer} mappában hozzon létre egy .env fájlt és töltse ki a következők szerint:
          \begin{lstlisting}
        AMQP_ADDR=amqp://szervercim:5672
        REDIS_ADDR=redis://:redisjelszo:szervercim:6379\end{lstlisting}

    \item A \emph{szervercim} helyére írja be a docker csomagot futtató szerver címét, a \emph{redisjelszo} helyére pedig a \textbf{\nameref{config}} részben definiált Redis jelszót.

    \item Indítsa el a bináris futtatható fájlt.
\end{enumerate}