\section{Szerkesztő}
\rhead{Szerkesztő}

\subsection{Technológiák}

A 3D szerkesztő elkészítése számos technikai és tervezési problémát vetett fel. Egy olyan szoftver
megalkotása volt a cél, amely elég egyszerű ahhoz, hogy bárki használhassa, viszont megfelelően
eszközdús ahhoz, hogy képes legyen bonyolult épületek tervezésére is. Ezen feltételek
megfogalmazása után arra jutottunk, hogy a 3D szerkesztőt egy web alapú alkalmazás keretein belül
fogjuk megvalósítani.

Egy ilyen szoftver esetében nagyon fontos a sebesség, ezért úgy döntöttünk, hogy a
böngészőkben használt JavaScript helyett a sokkal gyorsabb --- ámbár fiatalabb és éretlenebb ---
WebAssembly technológiát fogjuk használni. Mivel a WebAssembly önmagában csak egy utasításkészlet,
szükségünk volt egy programozási nyelvre, amit lehetséges WebAssembly utasításokra fordítani.
Erre a célra a Rust programozási nyelvet választottuk. Ennek a döntésnek több oka is van:

\begin{itemize}
      \item Gyorsaság: A Rust egy rendszerközeli nyelv, hasonlóan a C-hez vagy C++-hoz, így rengeteg
            olyan optimalizáció valósítható meg, ami magasabb szintű nyelvekben nem lehetséges.

      \item Fejlett WebAssembly eszközök: Annak ellenére, hogy a nyelv viszonylag fiatal, rengeteg
            kiváló eszköz és könyvtár áll rendelkezésre a WebAssembly programok fejlesztésének
            elősegítéséhez. Ezek közül talán a legfontosabb a wasm-bindgen könyvtár, amellyel triviális
            feladattá válik a böngésző által használt JavaScript kód és a WebAssemblyre átforduló Rust
            kód összekötése.

      \item Tapasztalat: A fejlesztői csapat két tagja már használta a Rust programozási nyelvet a
            múltban, így elkerülhettük az új technológiák tanulásából adódó problémákat.
\end{itemize}

\subsection{Működési elvek}

A használt technológiák kiválasztása után megkezdődött a szerkesztő működési elveinek felállítása.
Hamar eldöntöttük, hogy nagyrészt két, már létező szoftver működési elveit szeretnénk kiinduló
pontként használni.

Az egyik szoftver, a Blender egy általános rendeltetésű háromdimenziós modellező program.
Rendkívül sokoldalú és bonyolult, ezért csak néhány mechanizmust ültettünk át a saját
szerkesztőnkbe, például a kijelölés és a mozgatás módját.

A másik kiindulási pontként használt szoftver az ún. Hammer Editor. A program a Valve Software
által fejlesztett Source játékmotor pályaszerkesztő alkalmazása. A Hammer Editor egyik erőssége,
és egyben az a tulajdonsága amit mi is megvalósítottunk az, hogy benne bármilyen komplex beltéri
vagy kültéri jelenet lemodellezhető kisebb, egyszerűb alakzatok kombinálásával.

\subsection{Kommunikáció a böngészővel}

Problémák és kihívások nem csak a tervezési fázisban bukkantak fel; fejlesztés során is akadt
bőven. Az egyik legelső probléma, amivel fejlesztés közben találkoztunk, az a böngészővel való
kommunikáció kérdése. Mivel a JavaScript alapú felhasználói felület és a Rust alapú szerkesztő
valójában két független alkalmazás, ezért ki kellett találni egy módszert az információmegosztásra.
Az egyszerű függvényhívások két okból nem feleltek meg. Először is, a JavaScript és WebAssembly
kód natívan csak egész számok átadására képes, ami mi céljainkhoz kevés. Másodszor, mivel a
szerkesztő teljes mértékben WebAssembly memóriában él, nincs semmilyen JavaScript objektum, aminek
a függvényeit le lehetne hívni.

Az első problémánkra a wasm-bindgen nevű Rust könyvtár adott megoldást. A wasm-bindgen lehetségessé
teszi az egész számoknál komplexebb adatstruktúrák átültétését Rust kódból JavaScript kódba, és
fordítva. A második probléma megoldása kicsit nehezebbnek bizonyult. Végül egy olyan
modellt valósítottunk meg, amiben a két fél két különböző módszert használ a kommunikációhoz.
Amikor a böngésző kommunikál a szerkesztővel, azt üzenetekkel teszi. Ezek az üzenetek egyszerű
adatstruktúrák amik egy aszinkron csatornán keresztül jutnak el a szerkesztőhöz. Amikor az
információátadás fordítva történik, a helyzet egyszerűbb: a JavaScript oldal egyszerűen átad egy
anoním függvényt a Rust oldalnak, amit aztán a szerkesztő szükség esetén meghív.

\subsection{3D grafika}

A szerkesztő egyik legfontosabb feladata, hogy képes legyen a jelenetet kirajzolni a képernyőre.
Ezt teszik lehetővé a különböző grafikus API-k, mint például az OpenGL, Vulkan, vagy böngésző
esetén a WebGL. A Rust programozási nyelvhez rengeteg olyan könyvtár érhető el, amik ezeket az
API-kat elérhetővé teszik. Mi úgy döntöttünk, hogy a wgpu könyvtárat fogjuk használni. Ennek
legfőbb oka az, hogy egy letisztult, biztonságos réteget húz a régies WebGL interfészre.

\subsection{Parancs alapú szerkesztés}

Az előzetes tervezési fázisban számunkra fontos kikötés volt, hogy a szerkesztés közben
megvalósulhasson az ún. non-destruktív munkamenet. Ez azt jelenti, hogy bármit, amit a felhasználó
megtesz, azt vissza lehet vonni. Ahhoz, hogy ez lehetséges legyen, valamilyen formában el kell
tárolni a múltbeli szerkesztési lépéseket. Végül úgy döntöttünk, hogy az egész szerkesztőt ezen
alapelv köré építjük fel. Létrehoztunk egy speciális adatszerkezetet, amely az összes elvégezhető
szerkesztési lépést képes eltárolni. Amikor a felhasználó változtat valamit a jeleneten, egy
ilyen adatstruktúra (akció) kerül átadásra. Az végrehajtás után a szerkesztő felépíti az akció
ellentétét, az inverz akciót. Végül az inverz akció eltárolásra kerül egy ún. visszavonási veremben.
Amikor a felhasználó kiadja a visszavonási parancsot, a visszavonási verem tetején lévő akció
(ha létezik) újra végrehajtásra kerül, és belekerül egy másodlagos verembe, ami a visszavont
akciók inverzét tárolja. Ez azért szükséges, mert visszavonás mellet ismétlés is lehetséges.

\subsection{Külső erőforrások kezelése}

A tervezési fázisban hamar kiderült, hogy sok időt és energiát kell fektetnünk a szerkesztő által
használt külső erőforrások tárolásának és létrehozásának módjának meghatározásába. Arra jutottunk,
hogy ezen célok teljesítéséhez saját fájlformátumokat fogunk kifejleszteni. A fejlesztés során
felmerülő igények kielégítésére végül három új fájlformátum született meg: .ascn, .amdl, és .agzm.
Mindhárom fájlformátum bináris kódolású, előállításukat és beolvasásukat a bincode könyvtár
végzi. Az adatokat mindegyik formátum egymásba ágyazott struktúrákban tárolja.

\subsubsection{Jelenetfájl (.ascn)}

Az első általunk fejlesztett fájlformátum, a .ascn arra szolgál, hogy a szerkesztőben készített
3D jeleneteket tárolja. Az egész adatszerkezet egyetlen \emph{Scene} nevű struktúrában helyezkedik
el. Ennek a struktúrának két eleme van: \emph{camera} és \emph{world}. A \emph{Camera} struktúrában
a kamera pozíciója és forgása tárolódik, míg a \emph{World} struktúra kicsit bonyolultabb. Szintén
két eleme van: \emph{solids} és \emph{props}. Mindkét elem vektoros adatszerkezet, azaz
határozatlan mennyiségű homogén adatot tárol.

A solids elem \emph{Solid} típusú struktúrákat tárol. Egy Solid példány egy primitív szilárd
alakzatot modellez. Tartalmazza az alakzat csúcsait és lapjait, emellett a lapok textúráit.

A props elem \emph{Prop} típusú struktúrákat tárol. Egy Prop példány egy díszítőelem adatait
tárolja. Egy díszítőelemnek modell ID-je, pozíciója és forgása van.

\subsubsection{Modellfájl (.amdl)}

A .amdl fájlformátum egy tetszőlegesen komplex 3D modell tárolására szolgál, néhány korlátozással. A formátum például nem képes animációkat kódolni, hiszen a szerkesztő nem támogatja a mozgó tárgyak megjelenítését. A modellfájl gyökérstruktúrája \emph{Prop} névre hallgat. Két eleme van: \emph{bounding\_box} és \emph{meshes}. A \emph{bounding\_box} elem a modell köré írható, tengelyekhez igazított téglatestet tárolja. A \emph{meshes} elem vektoros adatszerkezet, ami a modellt felépítő, textúrázott almodelleket tárolja.

\subsubsection{Gizmofájl (.agzm)}

Az Archytex szerkesztő forráskódjában a gizmo szónak két jelentése is van: hivatkozhat az egér
általi mozgatást elősegítő alakzatokra, de arra a speciális 3D modellre is, aminek nincsenek
almodelljei és textúrái sem. A gizmofájl az utóbbi modellek tárolására szolgál. A három saját
fájlformátum közül ez a legegyszerűbb, hisz csak egyszerű geometriát kódol: csúcsokat, és az
azokból alkotott háromszögeket.