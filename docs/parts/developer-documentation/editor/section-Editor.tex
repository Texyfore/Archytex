\section{Szerkesztő}
\rhead{Szerkesztő}

\subsection{Technológiák}

A 3D szerkesztő elkészítése számos technikai és tervezési problémát vetett fel. Egy olyan szoftver
megalkotása volt a cél, amely elég egyszerű ahhoz, hogy bárki használhassa, viszont megfelelően
eszközdús ahhoz, hogy képes legyen bonyolult épületek tervezésére is. Ezen feltételek
megfogalmazása után arra jutottunk, hogy a 3D szerkesztőt egy web-alapú alkalmazás keretein belül
fogjuk megvalósítani.

Egy ilyen szoftver esetében nagyon fontos a sebesség, ezért úgy döntöttünk, hogy a
böngészőkben használt JavaScript helyett a sokkal gyorsabb --- ámbár fiatalabb és éretlenebb ---
WebAssembly technológiát fogjuk használni. Mivel a WebAssembly önmagában csak egy utasításkészlet,
szükségünk volt egy programozási nyelvre, amit lehetséges WebAssembly utasításokra fordítani.
Erre a célra a Rust programozási nyelvet választottuk. Ennek a döntésnek több oka is van:

\begin{itemize}
    \item Gyorsaság: A Rust egy rendszerközeli nyelv, hasonlóan a C-hez vagy C++-hoz, így rengeteg
    olyan optimalizáció valósítható meg, ami magasabb szintű nyelvekben nem lehetséges.

    \item Fejlett WebAssembly eszközök: Annak ellenére, hogy a nyelv viszonylag fiatal, rengeteg
    kiváló eszköz és könyvtár áll rendelkezésre a WebAssembly programok fejlesztésének
    elősegítéséhez. Ezek közül talán a legfontosabb a wasm-bindgen könyvtár, amellyel triviális
    feladattá válik a böngésző által használt JavaScript kód és a WebAssemblyre átforduló Rust
    kód összekötése.

    \item Tapasztalat: A fejlesztői csapat két tagja már használta a Rust programozási nyelvet a
    múltban, így elkerülhettük az új technológiák tanulásából adódó problémákat.
\end{itemize}

\subsection{Működési elvek}

A használt technológiák kiválasztása jután megkezdődött a szerkesztő működési elveinek felállítása.
Hamar eldöntöttük, hogy nagyrészt két már létező szoftver működési elveit szeretnénk kiinduló
pontként használni, amire aztán a saját ötleteinket építhetjük.

Az egyik szoftver, a Blender egy általános rendeltetésű háromdimenziós modellező program.
Rendkívül sokoldalú és bonyolult, ezért csak néhány mechanizmust ültettünk át a saját
szerkesztőnkbe, például a kijelölés és a mozgatás módját.


\subsection{Kommunikáció a böngészővel}
\subsection{3D grafika}
\subsection{Parancs alapú szerkesztés}
\subsection{Külső erőforrások kezelése}